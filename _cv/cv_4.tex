%%%%%%%%%%%%%%%%%%%%%%%%%%%%%%%%%%%%%%%%%
% Medium Length Professional CV
% LaTeX Template
% Version 2.0 (8/5/13)
%
% This template has been downloaded from:
% http://www.LaTeXTemplates.com
%
% Original author:
% Trey Hunner (http://www.treyhunner.com/)
%
% Important note:
% This template requires the resume.cls file to be in the same directory as the
% .tex file. The resume.cls file provides the resume style used for structuring the
% document.
%
%%%%%%%%%%%%%%%%%%%%%%%%%%%%%%%%%%%%%%%%%

%----------------------------------------------------------------------------------------
%	PACKAGES AND OTHER DOCUMENT CONFIGURATIONS
%----------------------------------------------------------------------------------------

\documentclass{resume} % Use the custom resume.cls style

\usepackage[left=0.75in,top=0.6in,right=0.75in,bottom=0.6in]{geometry} % Document margins
\usepackage{multicol}
\usepackage{bold-extra}
\usepackage{hyperref}
\hypersetup{
    colorlinks=true,
    linkcolor=blue,
    filecolor=magenta,
    urlcolor=blue,
}
\newcommand{\tab}[1]{\hspace{.2667\textwidth}\rlap{#1}}
\newcommand{\itab}[1]{\hspace{0em}\rlap{#1}}
\name{Seung Whan Chung} % Your name
%\address{502 E Michigan Ave. Apt 20, Urbana, IL 61801} % Your address
%\address{123 Pleasant Lane \\ City, State 12345} % Your secondary addess (optional)
\address{Computational Scientist \\ Center for Applied Scientific Computing \\ Lawrence Livermore National Laboratory}
\address{\href{mailto:chung28@llnl.gov}{chung28@llnl.gov} \\ \href{https://www.chung-research.com}{chung-research.com} \\ \href{https://www.linkedin.com/in/kevin-seung-whan-chung}{linkedin}} % Your phone number and email
%\address{https://-}

\begin{document}

%----------------------------------------------------------------------------------------
%	EDUCATION SECTION
%----------------------------------------------------------------------------------------

\begin{rSection}{Education}
{\bf University of Illinois at Urbana-Champaign} \hfill {\em January 2017 - August 2021}
\\ Ph.\ D in Theoretical and Applied Mechanics \hfill {GPA: 4.0/4.0}\\
{\bf University of Illinois at Urbana-Champaign} \hfill {\em August 2014 - December 2016}
\\ M.\ S in Theoretical and Applied Mechanics \hfill {GPA: 3.88/4.0}\\
{\bf Seoul National University} \hfill {\em March 2008 - February 2014}
\\ B.\ S. in Mechanical and Aerospace Engineering (Summa cum laude) \hfill {GPA: 3.96/4.3}
\end{rSection}

%----------------------------------------------------------------------------------------
%	RESEARCH INTERESTS
%----------------------------------------------------------------------------------------

%\begin{rSection}{Research Interests}
%Adjoint-based optimization, Chaotic dynamics, flow control, Compressible turbulence
%%Chaos, Control, Optimization, Adjoint, Nonlinear dynamics, Data-driven model-reduction
%\end{rSection}

%----------------------------------------------------------------------------------------
%	WORK EXPERIENCE SECTION
%----------------------------------------------------------------------------------------

\begin{rSection}{Research}
%\begin{rSubsection}{Multi-point penalty-based optimization for chaotic flow control}{}{}{}
%\item[] \textit{Graduate researcher} \hfill \textit{University of Illinois at Urbana-Champaign}
%\item[] Advisor: Prof. Jonathan B. Freund \hfill December 2019 - Present
%\item Quantified and analyzed optimization performance degradation in chaotic dynamical systems.
%\item Developed multi-point penalty-based optimization framework for non-convex optimization of chaotic flows.
%\item Demonstrated the method in various chaotic flow control optimizations,
%from 1D Kuramoto--Sivashinsky equation to 3D turbulent Kolmogorov flow.
%\item In preparation for a publication.
%\end{rSubsection}
%
%\begin{rSubsection}{Adjoint-based optimization for a supersonic jet noise}{}{}{}
%\item[] \textit{Graduate researcher} \hfill \textit{University of Illinois at Urbana-Champaign}
%\item[] Advisor: Prof. Jonathan B. Freund \hfill May 2017 - December 2019
%\item Implemented a compressible Mach-1.3 jet simulation,
%using a Fortran-based Navier-Stokes solver with energy-stable high-order finite-difference discretization.
%\item Verified turbulence development of the jet
%\item Implemented Ffowcs-Williams-Hawkings (FWH) solver to validate sound radiation of the jet
%\item Performed the jet noise control optimization using $10^4$ CPUs, and quantified optimization performance degradation in the chaotic turbulent jet.
%\end{rSubsection}
%
%\begin{rSubsection}{Sensitivity algorithm for particle-in-cell (PIC) plasma kinetics}{}{}{}
%\item[] \textit{Graduate researcher} \hfill \textit{Center for Exascale Simulation of Plasma-Coupled Combustion}
%\item[] Advisor: Prof. Jonathan B. Freund \hfill January 2015 - January 2017
%\item[] \textit{Student intern} \hfill \textit{Sandia National Laboratories}
%\item[] Mentor: Dr. Stephen D. Bond, Dr. Eric C. Cyr \hfill January 2017 - May 2017
%\smallskip
%%\item Developed a fortran-based 1D PIC-MCC code to carry out plasma kinetics simulations.
%\item Formulated discrete, particle-exact sensitivity in PIC simulation,
%and demonstrated sensitivity degradation due to chaotic particle dynamics.
%\item Participated in a 4-month student internship at Sandia National Laboratories for collaboration.
%\item Developed new particle-pdf sensitivity method
%which avoids the chaotic effect of particle dynamics.
%Published a peer-reviewed journal paper.
%\item Demonstrated the sensitivity algorithm
%for the sensitivity of Debye shielding response and sheath edge formation.
%\item Developed a Fortran-based 2D finite-volume Vlasov solver for validation of the new sensitivity algorithm.
%\end{rSubsection}

\begin{rSubsection}{Lawrence Livermore National Laboratory}{April 2024 - Present}{Computational Scientist}{Livermore, CA}
\item[]
\vspace{-20pt}
\end{rSubsection}

\begin{rSubsection}{Lawrence Livermore National Laboratory}{January 2023 - March 2024}{Postdoctoral Staff Member}{Livermore, CA}
\item Developed a scalable reduced order model with discontinuous Galerkin domain decomposition
\item Orchestrated the development of \texttt{pylibROM}, python interface for the library of reduced order modeling
\item Advised and mentored three student interns (Ping-Hsuan Tsai, Seung-Won Suh, Axel Larsson)
\end{rSubsection}

\begin{rSubsection}{University of Texas at Austin}{September 2021 - December 2022}{Postdoctoral Fellow \textnormal{(with Prof. R. Moser, Prof. L. Raja, Dr. T. Oliver)}}{Austin, TX}
%\item Developed a fortran-based 1D PIC-MCC code to carry out plasma kinetics simulations.
%\item Formulated discrete, particle-exact sensitivity in PIC simulation,
%and demonstrated sensitivity degradation due to chaotic particle dynamics.
%\item Participated in a 4-month student internship at Sandia National Laboratories for collaboration.
%\item Developed new particle-pdf sensitivity method
%which avoids the chaotic effect of particle dynamics.
%Published a peer-reviewed journal paper.
%\item Demonstrated the sensitivity algorithm
%for the sensitivity of Debye shielding response and sheath edge formation.
%\item Developed a Fortran-based 2D finite-volume Vlasov solver for validation of the new sensitivity algorithm.
\item Uncertainty quantification of electron-argon collision cross sections via Bayesian inference
\item Physics-based reduced-modeling of inductively-coupled argon plasma torch
\item Developed a discontinuous-Galerkin HPC solver for large-scale non-equilibrium plasma simulations
\end{rSubsection}

\begin{rSubsection}{University of Illinois at Urbana-Champaign}{January 2015 - August 2021}{Graduate Researcher \textnormal{(with Prof. Jonathan Freund)}}{Urbana, IL}
%\item Developed a fortran-based 1D PIC-MCC code to carry out plasma kinetics simulations.
%\item Formulated discrete, particle-exact sensitivity in PIC simulation,
%and demonstrated sensitivity degradation due to chaotic particle dynamics.
%\item Participated in a 4-month student internship at Sandia National Laboratories for collaboration.
%\item Developed new particle-pdf sensitivity method
%which avoids the chaotic effect of particle dynamics.
%Published a peer-reviewed journal paper.
%\item Demonstrated the sensitivity algorithm
%for the sensitivity of Debye shielding response and sheath edge formation.
%\item Developed a Fortran-based 2D finite-volume Vlasov solver for validation of the new sensitivity algorithm.
\item Developed multi-point penalty-based optimization framework for chaotic turbulent flows.
\item Implemented and validated turbulence statistics and sound radiation of a compressible Mach-1.3 jet.
%\item Developed a novel regular gradient computing method for chaotic particle plasma simulations.
\end{rSubsection}

\begin{rSubsection}{Sandia National Laboratories}{January 2017 - May 2017}{Student Intern \textnormal{(with Dr. Stephen D. Bond, Dr. Eric C. Cyr)}}{Albuquerque, NM}
%\item[] \textit{Student intern} \hfill \textit{}
%\item[]  \hfill
%\smallskip
%%\item Developed a fortran-based 1D PIC-MCC code to carry out plasma kinetics simulations.
%\item Formulated discrete, particle-exact sensitivity in PIC simulation,
%and demonstrated sensitivity degradation due to chaotic particle dynamics.
%\item Participated in a 4-month student internship at Sandia National Laboratories for collaboration.
\item Developed a novel regular gradient computing method for chaotic particle plasma simulations.
%Published a peer-reviewed journal paper.
\item Demonstrated gradient computation for Debye shielding response and sheath edge formation.
%\item Developed a Fortran-based 2D finite-volume Vlasov solver for validation of the new sensitivity algorithm.
\end{rSubsection}

\end{rSection}
%------------------------------------------------
%\clearpage
%----------------------------------------------------------------------------------------
%	TECHNICAL STRENGTHS SECTION
%----------------------------------------------------------------------------------------

\begin{rSection}{Skills}

\begin{tabular}{ @{} >{\bfseries}l @{\hspace{6ex}} l }
Computer Languages &  Python, C++, MATLAB, Fortran, pybind11 \\
Parallel Programming & MPI \\
Simulation Libraries & \href{https://mfem.org}{\texttt{MFEM}}, \href{https://www.librom.net/}{\texttt{libROM}}, \href{https://gmsh.info/}{\texttt{Gmsh}} \\
Scripting & Python, Bash, Flux \\
Version Control & Git, Docker \\
Documentation & \LaTeX, Vi/Vim, Mendeley \\
Visualization and I/O & PLOT3D, HDF5, Paraview \\
Presentation & Beamer, Keynote, Adobe Illustrator/Premiere \\
\end{tabular}

\end{rSection}

%----------------------------------------------------------------------------------------
\clearpage
\begin{rSection}{Publications}
\textbf{S.\ W.\ Chung}, Y.\ Choi, P.\ Roy, T.\ Moore, T.\ Roy, T.\ Lin, D.\ T.\ Nguyen, C.\ Hahn, E.\ B.\ Duoss \& S.\ E.\ Baker,
``Train small, model big: scalable physics simulators via reduced order modeling and domain decomposition,"
\textit{Computer Methods in Applied Mechanics and Engineering}, \textbf{427}, (2024).
\par
\textbf{S.\ W.\ Chung}, T.\ A.\ Oliver, L. Raja \& R.\ D.\ Moser,
``Characterization of uncertainties in electron-argon collision cross sections under statistical principles,"
\textit{Plasma Sources Science and Technology}, submitted, (2023).
\par
%\textbf{S.\ W.\ Chung} \& J.\ B.\ Freund,
%``Finding an optimal free-space flow control with multi-point penalty method,"
%\textit{In preparation}.
\par
\textbf{S.\ W.\ Chung} \& J.\ B.\ Freund,
``An optimization method for chaotic turbulent flows,"
\textit{Journal of Computational Physics}, \textbf{457}, (2022).
\par
\textbf{S.\ W.\ Chung}, S.\ D.\ Bond, E.\ C.\ Cyr, \& J.\ B.\ Freund,
``Regular sensitivity computation avoiding chaotic effects in particle-in-cell plasma methods,"
\textit{Journal of Computational Physics}, \textbf{400} (2020).
\end{rSection}

%----------------------------------------------------------------------------------------
%\vspace*{-30pt}
\begin{rSection}{Conference Talks}
\textbf{S.\ W.\ Chung}, Y.\ Choi, P.\ Roy, T.\ Roy, T.\ Moore, T.\ Lin \& S.\ E.\ Baker,
``Scalable physics-guided data-driven component model reduction for Stokes flow,"
\textit{NeurIPS 2023 Workshop on the Machine Learning and the Physical Sciences} (2023).
\par
P.-H.\ Tsai, \textbf{S.\ W.\ Chung}, D.\ Ghosh, J.\ Loffeld, Y.\ Choi \& J.\ L.\ Belof,
``Accelerating Kinetic Simulations of Electrostatic Plasmas with Reduced-Order Modeling,"
\textit{NeurIPS 2023 Workshop on the Machine Learning and the Physical Sciences} (2023).
\par
S.\ W.\ Suh, \textbf{S.\ W.\ Chung}, T.\ Bremer \& Y.\ Choi,
``Accelerating Flow Simulations using Online Dynamic Mode Decomposition,"
\textit{NeurIPS 2023 Workshop on the Machine Learning and the Physical Sciences} (2023).
\par

\textbf{S.\ W.\ Chung} \& J.\ B.\ Freund. "Finding an optimal flow control with multi-point penalty method,"
\textit{Bulletin of the American Physical Society}, \textbf{67} (2022).

\par
\textbf{S.\ W.\ Chung}, T.\ A.\ Oliver, L.\ L.\ Raja \& R.\ D.\ Moser,  ``Characterization of uncertainties in electron-argon collision cross sections under statistical principles,''
\textit{Bulletin of the American Physical Society}, \textbf{67} (2022).

\par
\textbf{S.\ W.\ Chung} \& J.\ B.\ Freund. "Multi-point penalty-based optimization for optimal control of chaotic turbulent flow,"
\textit{Bulletin of the American Physical Society}, \textbf{66} (2021).

\par
\textbf{S.\ W.\ Chung} \& J.\ B.\ Freund, ``Multi-point augmented Lagrangian optimization for chaotic flows,"
\textit{SIAM Conference on Computational Science and Engineering}, (2021).

\par
\textbf{S.\ W.\ Chung} \& J.\ B.\ Freund. "Multi-point augmented Lagrangian optimization for chaotic flows,"
\textit{Bulletin of the American Physical Society}, \textbf{65} (2020).

\par
\textbf{S.\ W.\ Chung} \& J.\ B.\ Freund,
``Adjoint­-based analysis of controllability of turbulent jet noise,"
\textit{Bulletin of the American Physical Society}, \textbf{64} (2019).

\par
\textbf{S.\ W.\ Chung}, S.\ D.\ Bond, E.\ C.\ Cyr, \& J.\ B.\ Freund,
``Regular sensitivity computation avoiding chaotic effects in particle-in-cell plasma methods,"
\textit{International Conference on Numerical Simulation of Plasmas}, (2019).

\par
\textbf{S.\ W.\ Chung}, S.\ D.\ Bond, E.\ C.\ Cyr, \& J.\ B.\ Freund, ``Sensitivity analysis in particle-in-cell methods,"
\textit{SIAM Conference on Computational Science and Engineering}, (2019).

\par
\textbf{S.\ W.\ Chung}, R.\ Vishnampet, D.\ Bodony, \& J.\ B.\ Freund, ``Adjoint-based sensitivity of jet noise to near-nozzle forcing,"
\textit{Bulletin of the American Physical Society}, \textbf{62} (2017).
\end{rSection}

%----------------------------------------------------------------------------------------
\clearpage
\begin{rSection}{Invited Talks}
\textbf{S.\ W.\ Chung}, \textit{FEM@LLNL Seminar}, Lawrence Livermore National Laboratory, (2024).
\par
J.\ B.\ Freund \& \textbf{S.\ W.\ Chung}, Lawrence Livermore National Laboratory, (2021).
\par
\textbf{S.\ W.\ Chung}, \textit{Fluid Mechanics Seminar}, University of Illinois at Urbana-Champaign, (2020).
\par
\textbf{S.\ W.\ Chung}, Sandia National Laboratories, (2017).
\end{rSection}

%----------------------------------------------------------------------------------------
\begin{rSection}{Journal Referee}
Journal of Fluid Mechanics (2022-present)
%\par
\end{rSection}

%------------------------------------------------

\begin{rSection}{Research tools developed}
\begin{rSubsection}{\texttt{scaleupROM}: Scalable Physics-guided Reduced Order Model}{}
{ }{\url{https://github.com/LLNL/scaleupROM}}
\item A data-driven discontinuous Galerkin FEM for general PDE systems
based upon \href{https://mfem.org}{\texttt{MFEM}} and \href{https://www.librom.net}{\texttt{libROM}}
\item Developed and demonstrated the framework for various physics
\end{rSubsection}
\begin{rSubsection}{\texttt{pylibROM}: python interface for libROM}{}
{ }{\url{https://github.com/LLNL/pylibROM}}
\item Implemented efficient python interface for \href{https://www.librom.net}{\texttt{libROM}} classes
\item Demonstrated examples of DMD and projection-based ROM for various physics systems
\end{rSubsection}
\begin{rSubsection}{\texttt{libROM}: Library for Reduced Order Models}{}
{ }{\url{https://www.librom.net/}}
\item Implemented and maintained Docker container and CI workflow
\end{rSubsection}
\begin{rSubsection}{\texttt{TPS}: Torch Plasma Simulator}{}
{with M.\ Bolinches, T.\ Oliver, K.\ Schulz, R.\ Moser}{\url{https://github.com/pecos/tps}}
\item A discontinuous-Galerkin multi-physics application to support a plasma torch prediction,
implmented upon a gpu-enabled finite-element library (\href{https://mfem.org}{\texttt{MFEM}})
\item Formulated and implmented a two-temperature non-equilibrium reacting flow solver
\end{rSubsection}
\begin{rSubsection}{\texttt{magudi}: Dual-consistent, Discrete-exact Adjoint solver for Compressible Flows}{}
{with R.\ Vishnampet, J.\ B.\ Freund}{\url{https://github.com/dreamer2368/magudi}}
\item A Fortran-based compressible flow solver, equipped with discrete-exact adjoint-based gradient.
\item Incorporated a Python-based framework for multi-point penalty-based optimization capability.
\end{rSubsection}
\begin{rSubsection}{\texttt{torch1d}: one-dimensional reduced-model for inductively-coupled plasma torch}{}
{with T.\ Oliver, R.\ Moser}{\url{https://github.com/pecos/torch1d}}
\item A Python-based finite-difference solver for a one-dimensional reduced torch model
\item Supports low-Mach limit formulation for two-temperature non-equilibrium plasma
\end{rSubsection}
\begin{rSubsection}{\texttt{adjoint playground}: Adjoint, penalty-based optimization for chaotic flow controls}{}
{with J.\ B.\ Freund}{\em Available upon request}
\item A MATLAB-based penalty-based optimization framework for various chaotic dynamical systems.
\item Provides a discrete-exact adjoint gradient for semi-implicit Runge-Kutta 4th-order time integrator.
\end{rSubsection}
\begin{rSubsection}{\texttt{PASS}: Particle Adjoint Sensitivity Sandbox}{}
{with J.\ B.\ Freund}{\url{https://github.com/dreamer2368/PASS}}
\item A Fortran-based 1D Particle-in-Cell code for plasma kinetics, with adjoint gradient capability
%\item Particle-exact/particle-pdf sensitivity solver
\end{rSubsection}
\end{rSection}

%\clearpage

\begin{rSection}{Teaching}

%------------------------------------------------

\begin{rSubsection}{TAM 210/211: Statics}{Spring 2020}{Teaching Assistant}{University of Illinois at Urbana-Champaign}
\item Ranked as Excellent in the list of Spring 2020 semester.
\item Conducted discussion sessions (1 time/wk) for 27 students.
\item Prepared in-depth solution procedures.
\item Provided extended office hours: 6 hrs/wk
\end{rSubsection}

%------------------------------------------------

\end{rSection}

%	EXAMPLE SECTION
%----------------------------------------------------------------------------------------

%------------------------------------------------

\begin{rSection}{Awards/Fellowships}

{\bf Jeong-Song Fellowship}\hfill 2014 - 2016\\
{\it Jeong-Song Cultural Foundation, Korea}\hfill \$110,000

{\bf Honor Graduation Award}\hfill 2014\\
{\it Seoul National University}\hfill Ranked 5 of 139 (summa cum laude)

{\bf Presidential Science Fellowship}\hfill 2008 - 2014\\
{\it M. B. Lee, the President of Republic of Korea}\hfill \$40,000

\end{rSection}

%----------------------------------------------------------------------------------------
\begin{rSection}{Graduate Courses}
\hspace{-.7cm}
\begin{tabular}{p{4.5cm}p{5cm}l}
\textbf{Fluid Mechanics} & \textbf{Computational Methods} & \textbf{Applied Mechanics}\\
Inviscid Flow & Computational Mechanics & Control System Theory \& Design \\
Viscous Flow & Uncertainty Quantification &  Solid Mechanics I \\
Instability and Transition & Asymptotic Method & Combustion Fundamentals \\
Turbulence & Mathematical Methods II & Non-Newtonian Fluid Mechanics \& Rheology
\end{tabular}

\end{rSection}

%\clearpage
%
%%------------------------------------------------
%
%\begin{rSection}{Work Experience}
%
%\begin{rSubsection}{United States Army}{October 2009 - August 2011}{Sergeant}{HHC, 302D BSB, 1HBCT,  2ID, 8th Army}
%\item Served as a Korean Augmentation To the United States Army (KATUSA).
%\item Worked as a human resources specialist (42A) in the Battalion S-1.
%\item Finished Warrior Leadership Course (WLC) to be a non-commisioned officer (NCO).
%\item As a Sr.\ KATUSA, took in charge of 20 KATUSAs in the company,
%and facilitated communication between the United States Army and the Republic of Korea Army.
%\item Awarded Army Achievement Medal (AAM) and Army Commendation Medal (ARCOM) for the extraordinary leadership during the service.
%\end{rSubsection}
%
%%------------------------------------------------
%
%\end{rSection}

\end{document}
