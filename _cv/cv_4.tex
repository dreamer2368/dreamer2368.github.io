%%%%%%%%%%%%%%%%%%%%%%%%%%%%%%%%%%%%%%%%%
% Medium Length Professional CV
% LaTeX Template
% Version 2.0 (8/5/13)
%
% This template has been downloaded from:
% http://www.LaTeXTemplates.com
%
% Original author:
% Trey Hunner (http://www.treyhunner.com/)
%
% Important note:
% This template requires the resume.cls file to be in the same directory as the
% .tex file. The resume.cls file provides the resume style used for structuring the
% document.
%
%%%%%%%%%%%%%%%%%%%%%%%%%%%%%%%%%%%%%%%%%

%----------------------------------------------------------------------------------------
%	PACKAGES AND OTHER DOCUMENT CONFIGURATIONS
%----------------------------------------------------------------------------------------

\documentclass{resume} % Use the custom resume.cls style

\usepackage[left=0.75in,top=0.6in,right=0.75in,bottom=0.6in]{geometry} % Document margins
\usepackage{multicol}
\usepackage{bold-extra}
\usepackage{hyperref}
\hypersetup{
    colorlinks=true,
    linkcolor=blue,
    filecolor=magenta,      
    urlcolor=blue,
}
\newcommand{\tab}[1]{\hspace{.2667\textwidth}\rlap{#1}}
\newcommand{\itab}[1]{\hspace{0em}\rlap{#1}}
\name{Seung Whan Chung} % Your name
%\address{502 E Michigan Ave. Apt 20, Urbana, IL 61801} % Your address
%\address{123 Pleasant Lane \\ City, State 12345} % Your secondary addess (optional)
\address{PhD Student \\ Theoretical and Applied Mechanics \\ University of Illinois at Urbana-Champaign}
\address{(+1)~217~417~1921 \\ \href{mailto:schung58@illinois.edu}{schung58@illinois.edu} \\ \href{https://www.chung-research.com}{chung-research.com} \\ \href{https://www.linkedin.com/in/kevin-seung-whan-chung}{linkedin}} % Your phone number and email
%\address{https://-}

\begin{document}

%----------------------------------------------------------------------------------------
%	EDUCATION SECTION
%----------------------------------------------------------------------------------------

\begin{rSection}{Education}
{\bf University of Illinois at Urbana-Champaign} \hfill {\em January 2017 - Expected: August 2021} 
\\ Ph.\ D in Theoretical and Applied Mechanics (candidate) \hfill {GPA: 4.0/4.0}\\
{\bf University of Illinois at Urbana-Champaign} \hfill {\em August 2014 - December 2016} 
\\ M.\ S in Theoretical and Applied Mechanics \hfill {GPA: 3.88/4.0}\\
{\bf Seoul National University} \hfill {\em March 2008 - February 2014} 
\\ B.\ S. in Mechanical and Aerospace Engineering \hfill {GPA: 3.96/4.3}
\end{rSection}

%----------------------------------------------------------------------------------------
%	RESEARCH INTERESTS
%----------------------------------------------------------------------------------------

%\begin{rSection}{Research Interests}
%Adjoint-based optimization, Chaotic dynamics, flow control, Compressible turbulence
%%Chaos, Control, Optimization, Adjoint, Nonlinear dynamics, Data-driven model-reduction
%\end{rSection}

%----------------------------------------------------------------------------------------
\begin{rSection}{Publications}
\par
\textbf{S.\ W.\ Chung} \& J.\ B.\ Freund,
``A gradient-based optimization framework for optimal control of chaotic turbulent flows,"
\textit{In preparation}.
\par
\textbf{S.\ W.\ Chung}, S.\ D.\ Bond, E.\ C.\ Cyr, \& J.\ B.\ Freund,
``Regular sensitivity computation avoiding chaotic effects in particle-in-cell plasma methods,"
\textit{Journal of Computational Physics}, \textbf{400} (2020).
\end{rSection}

%----------------------------------------------------------------------------------------
\begin{rSection}{Conference Talks}
\textbf{S.\ W.\ Chung} \& J.\ B.\ Freund, ``Multi-point augmented Lagrangian optimization for chaotic flows,"
\textit{SIAM Conference on Computational Science and Engineering}, (2021).

\par
\textbf{S.\ W.\ Chung} \& J.\ B.\ Freund. "Multi-point augmented Lagrangian optimization for chaotic flows,"
\textit{Bulletin of the American Physical Society}, \textbf{65} (2020).

\par
\textbf{S.\ W.\ Chung} \& J.\ B.\ Freund,
``Adjoint­-based analysis of controllability of turbulent jet noise,"
\textit{Bulletin of the American Physical Society}, \textbf{64} (2019).

\par
\textbf{S.\ W.\ Chung}, S.\ D.\ Bond, E.\ C.\ Cyr, \& J.\ B.\ Freund,
``Regular sensitivity computation avoiding chaotic effects in particle-in-cell plasma methods,"
\textit{International Conference on Numerical Simulation of Plasmas}, (2019).

\par
\textbf{S.\ W.\ Chung}, S.\ D.\ Bond, E.\ C.\ Cyr, \& J.\ B.\ Freund, ``Sensitivity analysis in particle-in-cell methods,"
\textit{SIAM Conference on Computational Science and Engineering}, (2019).

\par
\textbf{S.\ W.\ Chung}, R.\ Vishnampet, D.\ Bodony, \& J.\ B.\ Freund, ``Adjoint-based sensitivity of jet noise to near-nozzle forcing,"
\textit{Bulletin of the American Physical Society}, \textbf{62} (2017).
\end{rSection}

%----------------------------------------------------------------------------------------
\begin{rSection}{Invited Talks}
J.\ B.\ Freund \& \textbf{S.\ W.\ Chung}, Lawrence Livermore National Laboratory, (2021).
\par
\textbf{S.\ W.\ Chung}, \textit{Fluid Mechanics Seminar}, University of Illinois at Urbana-Champaign, (2020).
\par
\textbf{S.\ W.\ Chung}, Sandia National Laboratories, (2017).
\end{rSection}

%------------------------------------------------

\begin{rSection}{Research tools developed}
\begin{rSubsection}{\texttt{PASS}: Particle Adjoint Sensitivity Sandbox}{}
{with J.\ B.\ Freund}{\url{https://github.com/dreamer2368/PASS}}
\item A Fortran-based 1D Particle-in-Cell -- Monte-Carlo-Collision code for plasma kinetics simulations.
\item Particle-exact/particle-pdf sensitivity solver
\end{rSubsection}
\clearpage
\begin{rSubsection}{\texttt{magudi}: Dual-consistent, Discrete-exact Adjoint solver for Compressible Flows}{}
{with R.\ Vishnampet, J.\ B.\ Freund}{\url{https://bitbucket.org/xpacc-dev/magudi/}}
\item Created verification cases to ensure discrete-exactness.
\item Developed a Python-based Bash/Flux-script generator for large-scale gradient-based optimization.
\item Incorporated multi-point penalty-based optimization framework for chaotic dynamical systems.
\end{rSubsection}
\begin{rSubsection}{\texttt{adjoint playground}: Adjoint, penalty-based optimization for chaotic flow controls}{}
{with J.\ B.\ Freund}{\em Available upon request}
\item A MATLAB-based penalty-based optimization framework for various chaotic dynamical systems.
\item Provides a discrete-exact adjoint gradient for semi-implicit Runge-Kutta 4th-order time integrator.
\end{rSubsection}
\end{rSection}

%----------------------------------------------------------------------------------------
%	WORK EXPERIENCE SECTION
%----------------------------------------------------------------------------------------

\begin{rSection}{Research}
\begin{rSubsection}{Multi-point penalty-based optimization for chaotic flow control}{}{}{}
\item[] \textit{Graduate researcher} \hfill \textit{University of Illinois at Urbana-Champaign}
\item[] Advisor: Prof. Jonathan B. Freund \hfill December 2019 - Present
\item Quantified and analyzed optimization performance degradation in chaotic dynamical systems.
\item Developed multi-point penalty-based optimization framework for non-convex optimization of chaotic flows.
\item Demonstrated the method in various chaotic flow control optimizations,
from 1D Kuramoto--Sivashinsky equation to 3D turbulent Kolmogorov flow.
\item In preparation for a publication.
\end{rSubsection}

\begin{rSubsection}{Adjoint-based optimization for a supersonic jet noise}{}{}{}
\item[] \textit{Graduate researcher} \hfill \textit{University of Illinois at Urbana-Champaign}
\item[] Advisor: Prof. Jonathan B. Freund \hfill May 2017 - December 2019
\item Implemented a compressible Mach-1.3 jet simulation,
using a Fortran-based Navier-Stokes solver with energy-stable high-order finite-difference discretization.
\item Verified turbulence development of the jet
\item Implemented Ffowcs-Williams-Hawkings (FWH) solver to validate sound radiation of the jet
\item Performed the jet noise control optimization using $10^4$ CPUs, and quantified optimization performance degradation in the chaotic turbulent jet.
\end{rSubsection}

\begin{rSubsection}{Sensitivity algorithm for particle-in-cell (PIC) plasma kinetics}{}{}{}
\item[] \textit{Graduate researcher} \hfill \textit{Center for Exascale Simulation of Plasma-Coupled Combustion}
\item[] Advisor: Prof. Jonathan B. Freund \hfill January 2015 - January 2017
\item[] \textit{Student intern} \hfill \textit{Sandia National Laboratories}
\item[] Mentor: Dr. Stephen D. Bond, Dr. Eric C. Cyr \hfill January 2017 - May 2017
\smallskip
%\item Developed a fortran-based 1D PIC-MCC code to carry out plasma kinetics simulations.
\item Formulated discrete, particle-exact sensitivity in PIC simulation,
and demonstrated sensitivity degradation due to chaotic particle dynamics.
\item Participated in a 4-month student internship at Sandia National Laboratories for collaboration.
\item Developed new particle-pdf sensitivity method
which avoids the chaotic effect of particle dynamics.
Published a peer-reviewed journal paper.
\item Demonstrated the sensitivity algorithm
for the sensitivity of Debye shielding response and sheath edge formation.
\item Developed a Fortran-based 2D finite-volume Vlasov solver for validation of the new sensitivity algorithm.
\end{rSubsection}

\end{rSection}
%------------------------------------------------
\clearpage
\begin{rSection}{Teaching}

%------------------------------------------------

\begin{rSubsection}{TAM 210/211: Statics}{Spring 2020}{Teaching Assistant}{University of Illinois at Urbana-Champaign}
\item Ranked as Excellent in the list of Spring 2020 semester.
\item Conducted discussion sessions (1 time/wk) for 27 students.
\item Prepared in-depth solution procedures.
\item Provided extended office hours: 6 hrs/wk
\end{rSubsection}

%------------------------------------------------

\end{rSection}

%	EXAMPLE SECTION
%----------------------------------------------------------------------------------------

%------------------------------------------------

\begin{rSection}{Awards/Fellowships}

{\bf Jeong-Song Fellowship}\hfill 2014 - 2016\\
{\it Jeong-Song Cultural Foundation, Korea}\hfill \$110,000

{\bf Honor Graduation Award}\hfill 2014\\
{\it Seoul National University}\hfill Ranked 5 of 139 (summa cum laude)

{\bf Presidential Science Fellowship}\hfill 2008 - 2014\\
{\it M. B. Lee, the President of Republic of Korea}\hfill \$40,000

\end{rSection}

%----------------------------------------------------------------------------------------
\begin{rSection}{Graduate Courses}
\hspace{-.7cm}
\begin{tabular}{p{4.5cm}p{5cm}l}
\textbf{Fluid Mechanics} & \textbf{Computational Methods} & \textbf{Applied Mechanics}\\
Inviscid Flow & Computational Mechanics & Control System Theory \& Design \\
Viscous Flow & Uncertainty Quantification &  Solid Mechanics I \\
Instability and Transition & Asymptotic Method & Combustion Fundamentals \\
Turbulence & Mathematical Methods II & Non-Newtonian Fluid Mechanics \& Rheology
\end{tabular}

\end{rSection}

%----------------------------------------------------------------------------------------
%	TECHNICAL STRENGTHS SECTION
%----------------------------------------------------------------------------------------

\begin{rSection}{Skills}

\begin{tabular}{ @{} >{\bfseries}l @{\hspace{6ex}} l }
Computer Languages &  Fortran, MATLAB, Python\\%, C/C++\\
Parallel Programming & MPI \\
Scripting & Python, Bash, Flux \\
Compiling & Make, CMake \\
Documentation & \LaTeX, Vi/Vim, Mendeley \\
Visualization and I/O & PLOT3D, Paraview \\
Presentation & Beamer, Keynote, Adobe Illustrator/Premiere \\
\end{tabular}

\end{rSection}

%\clearpage
%
%%------------------------------------------------
%
%\begin{rSection}{Work Experience}
%
%\begin{rSubsection}{United States Army}{October 2009 - August 2011}{Sergeant}{HHC, 302D BSB, 1HBCT,  2ID, 8th Army}
%\item Served as a Korean Augmentation To the United States Army (KATUSA).
%\item Worked as a human resources specialist (42A) in the Battalion S-1.
%\item Finished Warrior Leadership Course (WLC) to be a non-commisioned officer (NCO).
%\item As a Sr.\ KATUSA, took in charge of 20 KATUSAs in the company,
%and facilitated communication between the United States Army and the Republic of Korea Army.
%\item Awarded Army Achievement Medal (AAM) and Army Commendation Medal (ARCOM) for the extraordinary leadership during the service.
%\end{rSubsection}
%
%%------------------------------------------------
%
%\end{rSection}

\end{document}
